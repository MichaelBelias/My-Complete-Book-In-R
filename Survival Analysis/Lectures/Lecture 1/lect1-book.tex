\documentclass[11pt,slidesonly,semrot,portrait,palatino]{book}
\usepackage{amssymb}
\usepackage{amsmath}
\usepackage{epsfig}

\newcommand{\simdot}{\stackrel{\cdot}{\sim}}
\newcommand{\bfbeta}{{\mbox{\boldmath$\beta$}}}
\newcommand{\bfep}{{\mbox{\boldmath$\epsilon$}}}
\newcommand{\bhat}{\hat{\beta}}
\newcommand{\btilde}{\tilde{\mbox{\boldmath$\beta$}}}
\newcommand{\bfmu}{{\mbox{\boldmath$\mu$}}}
\newcommand{\Var}{{\rm Var}}
\newcommand{\Cov}{{\rm Cov}}
\newcommand{\trt}{{\rm trt}}
\newcommand{\pr}{{\rm pr}}
\newcommand{\age}{{\rm age}}
\newcommand{\Sin}{\sum_{i=1}^N}
\newcommand{\Sjn}{\sum_{j=1}^N}
\newcommand{\ui}{{\bf u}_i}
\newcommand{\uj}{{\bf u}_j}
\newcommand{\bfx}{{\mbox{{\bf x}}}}
\newcommand{\bfp}{{\mbox{{\bf p}}}}
\newcommand{\hbfp}{\widehat{\mbox{{\bf p}}}}
\newcommand{\bfy}{{\mbox{{\bf y}}}}
\newcommand{\bfY}{{\mbox{{\bf Y}}}}
\newcommand{\bfZ}{{\mbox{{\bf Z}}}}
\newcommand{\bfa}{{\mbox{{\bf a}}}}
\newcommand{\bfb}{{\mbox{{\bf b}}}}
\newcommand{\bfg}{{\mbox{{\bf g}}}}
\newcommand{\bfU}{{\bf U}}
\newcommand{\bfu}{{\mbox{{\bf u}}}}
\newcommand{\bfz}{{\mbox{{\bf z}}}}
\newcommand{\logit}{{\mbox{{logit}}}}
\newcommand{\bfzero}{{\mbox{{\bf 0}}}}
\newcommand{\hbeta}{{\widehat \beta}}
\newcommand{\heta}{{\widehat \eta}}
\newcommand{\hsigma}{{\widehat \sigma}}
\newcommand{\hmu}{{\widehat \mu}}
\newcommand{\hpi}{{\widehat \pi}}
\newcommand{\cI}{{\cal I}}
\newcommand{\bsigma}{{\bar \sigma}}
\newcommand{\brho}{{\bar \rho}}
\newcommand{\bx}{ {\bar {x} } }
\newcommand{\bY}{ {\bar {Y} } }
\newcommand{\hY}{ {\widehat {Y} } }
\newcommand{\hp}{ {\widehat {p} } }
\newcommand{\hVar}{ {\widehat {Var} } }

%\setlength{\baselineskip}{2.5em}
\begin{document}
\setlength{\evensidemargin}{0in}
\setlength{\oddsidemargin}{.5in}
\setlength{\topmargin}{0in}
\setlength{\textwidth}{6in}
\setlength{\textheight}{9in}
%\LARGE
\setcounter{page}{15} \setcounter{chapter}{1}
%%%%%%%%%%%%%%%%%%%%%%%%%%%%%%%%%%%%%%%%%%%%%%%%%%%%%%%%%%%%%%%%%%%%%%%
\chapter{Survival Analysis: Introduction}
Survival Analysis typically focuses on {\bf time to event} data.
In the most general sense, it consists of techniques for
positive-valued random variables, such as

\begin{itemize}
\item time to death
\item time to onset (or relapse) of a disease
\item length of stay in a hospital
\item duration of a strike
\item money paid by health insurance
\item viral load measurements
\item time to finishing a doctoral dissertation!
\end{itemize}
{\bf Kinds of survival studies include:}
    \begin{itemize}
    \item clinical trials
    \item prospective cohort studies
    \item retrospective cohort studies
    \end{itemize}
\newpage
\noindent Typically, survival data are not fully observed, but
rather are {\em censored}.\\[2ex]
%%%%%%%%%%%%%%%%%%%%%%%%%%%%%%%%%%%%%%%%%%%%%%%%%%%%%%%%%%%
{\bf In this course, we will:
\begin{itemize}
\item describe survival data
\item compare survival of several groups
\item explain survival with covariates
\item design studies with survival endpoints
\end{itemize}}
\noindent
{\bf Some useful references:}
%\vspace{-0.2in}
\begin{itemize}
\item Collett: {\em Modelling Survival Data in Medical Research}
\item Cox and Oakes: {\em Analysis of Survival Data}
\item Kleinbaum: {\em Survival Analysis:  A self-learning text}
\item Klein \& Moeschberger: {\em Survival Analysis:  Techniques for
censored and truncated data}
\item Cantor: {\em Extending SAS Survival Analysis Techniques for
Medical Research}
\item Allison: {\em Survival Analysis Using the SAS System}
\end{itemize}
%%%%%%%%%%%%%%%%%%%%%%%%%%%%%%%%%%%%%%%%%%%%%%%%%%%%%%%%%%%
%\LARGE
{\bf Some Definitions and notation}
\\[2ex]
{\bf Failure time random variables} are always {\bf non-negative}.
That is, if we denote the failure time by $T$, then $T \ge 0$.
\\[2ex]
$T$ can either be {\bf discrete} (taking a finite set of values,
e.g. $a_1,a_2,\ldots, a_n$) or {\bf continuous} (defined on
$(0, \infty))$.
\\[2ex]
A random variable $X$ is called a {\bf censored failure time random
variable} if $X = \min(T,U)$, where $U$ is a non-negative censoring
variable.

\section{Failure random variables}
In order to define a failure time random variable, we need:
\begin{itemize}
\item[(1)]  an unambiguous {\bf time origin} \\
(e.g. randomization to clinical trial, purchase of car)
\item[(2)]  a {\bf time scale} \\
(e.g. real time (days, years), mileage of a car)
\item[(3)]  definition of the {\bf event}
(e.g. death, need a new car transmission)
\end{itemize}
%%%%%%%%%%%%%%%%%%%%%%%%%%%%%%%%%%%%%%%%%%%%%%%%%%%%%%%%%%%
\newpage
\begin{figure}[h]
\begin{center}
\begin{picture}(210,220)(-10,0)
\thicklines
\put(0,220){\line(0,-1){170}}
%\put(0,230){\line(1,0){125}}
%\put(125,228){\bf X}
\put(10,215){\line(1,0){45}}
\put(55,213){\bf X}
\put(20,190){\line(1,0){155}}
\put(175,190){\circle*{5}}
\put(25,165){\line(1,0){10}}
\put(35,165){\circle*{5}}
\put(30,140){\line(1,0){100}}
\put(130,137){\bf X}
\put(40,115){\line(1,0){135}}
\put(175,115){\circle*{5}}
\put(45,90){\line(1,0){15}}
\put(75,87){\bf X}
\put(50,65){\line(1,0){80}}
\put(130,65){\circle*{5}}
\put(175,220){\line(0,-1){170}}
\put(-10,40){study}
\put(-10,33){opens}
\put(165,40){study}
\put(165,33){closes}
\put(4,25){\circle*{10}}
\put(8,23 ){  = censored observation}
\put(0,12){{\bf X} = event}
\end{picture}
\caption{Illustration of survival data}
\end{center}
\end{figure}
The illustration of survival data on the previous page shows
several features which are typically encountered in analysis
of survival data:

\begin{itemize}
\item individuals do not all enter the study at the same time
\item when the study ends, some individuals still haven't had
the event yet
\item other individuals drop out or get lost in the middle of
the study, and all we know about them is the last time they
were still ``free'' of the event
\end{itemize}
The first feature is referred to as {\bf ``staggered entry''}
\\[2ex]
The last two features relate to {\bf ``censoring''} of the
failure time events.
%%%%%%%%%%%%%%%%%%%%%%%%%%%%%%%%%%%%%%%%%%%%%%%%%%%%%%%%%%%
\section{Types of censoring:}
\begin{itemize}
\item \fbox{\bf Right-censoring}:\\[3ex]
only the r.v. $X_i=\min(T_i,U_i)$ is observed due to
\begin{itemize}
\item loss to follow-up
\item drop-out
\item study termination
\end{itemize}

We call this right-censoring because the true unobserved event
is to the right of our censoring time; i.e., all we know is
that the event has not happened at the end of follow-up.
\\[2ex]
In addition to observing $X_i$, we also get to see the {\bf failure indicator}:
\[\delta_i=\left\{\begin{array}{ccc}
1 & \mbox{if} & T_i\le U_i \\
0 & \mbox{if} & T_i>U_i \end{array} \right. \]

Some software packages instead assume we have a \\
{\bf censoring indicator}:
\[c_i=\left\{\begin{array}{ccc}
0 & \mbox{if} & T_i\le U_i \\
1 & \mbox{if} & T_i>U_i \end{array} \right. \]
Right-censoring is the most common type of censoring \\
assumption we will deal with in survival analysis.
%%%%%%%%%%%%%%%%%%%%%%%%%%%%%%%%%%%%%%%%%%%%%%%%%%%%%%%%%%%
\item \fbox{\bf Left-censoring}\\[3ex]
Can only observe $Y_i=\max(T_i,U_i)$ and the failure indicators:
\[\epsilon_i=\left\{\begin{array}{ccc}
1 & \mbox{if} & U_i\le T_i \\
0 & \mbox{if} & U_i>T_i \end{array} \right. \]

e.g. In studies of time to HIV seroconversion, some of the enrolled
subjects have already seroconverted at entry into the study - they are
left-censored.
\item  \fbox{\bf Interval-censoring}\\[3ex]
Observe $(L_i,R_i)$ where  $T_i\in (L_i,R_i)$\\

ex \#1: Time to prostate cancer, observe longitudinal PSA measurements

ex \#2: Time to undetectable viral load in AIDS studies, based on
measurements of viral load taken at each clinic visit
\end{itemize}
%%%%%%%%%%%%%%%%%%%%%%%%%%%%%%%%%%%%%%%%%%%%%%%%%%%%%%%%%%%
\subsection{Independent versus informative censoring}

\begin{itemize}
\item We say censoring is {\bf independent} (non-informative)
if $U_i$ is independent of $T_i$.
\begin{itemize}
\item {\bf ex.1} If $U_i$ is the planned end of the study (say, 2
years after the study opens), then it is usually independent of the
event times

\item {\bf ex.2} If $U_i$ is the time that a patient drops out of the study
because they've gotten much sicker and/or had to discontinue taking
the study treatment, then $U_i$ and $T_i$ are probably not independent
\end{itemize}
{\bf An individual censored at $U$ should be representative of all subjects
who survive to $U$.}
\\[2ex]
This means that censoring at $U$ {\em could} depend
on prognostic characteristics measured at baseline, but that among
all those with the same baseline characteristics, the probability
of censoring prior to or at time $U$ should be the same.

\item Censoring is considered {\bf informative} if the
distribution of $U_i$ contains any information about the parameters
characterizing the distribution of $T_i$.
\end{itemize}
%%%%%%%%%%%%%%%%%%%%%%%%%%%%%%%%%%%%%%%%%%%%%%%%%%%%%%%%%%%
Suppose we have a sample of observations on $n$ people:
$$(T_1,U_1), (T_2,U_2),..., (T_n,U_n)$$


There are three main types of censoring times:
\begin{itemize}
\item {\bf Type I:}  All the $U_i$'s are the same \\
e.g. animal studies, all animals sacrificed after 2 years \\
\item {\bf Type II:}  $U_i = T_{(r)} $, the time of the $r$th failure.\\
e.g. animal studies, stop when 4/6 have tumors\\
\item  {\bf Random:} the $U_i$'s are random variables, $\delta_i$'s are
failure indicators:
\[\delta_i=\left\{\begin{array}{ccc}
1 & \mbox{if} & T_i\le U_i\\
0 & \mbox{if} & T_i>U_i
\end{array} \right. \]
\end{itemize}
%%%%%%%%%%%%%%%%%%%%%%%%%%%%%%%%%%%%%%%%%%%%%%%%%%%%%%%%%%%
\section{Some example datasets}
\subsection{Duration of nursing home stay}
(Morris et al., {\em Case Studies in Biometry}, Ch 12)
\\[2ex]
The National Center for Health Services Research studied 36
for-profit nursing homes to assess the effects of
different financial incentives on length of stay.  ``Treated"
nursing homes received higher per diems for Medicaid patients,
and bonuses for improving a patient's health and sending them home.
\\[2ex]
Study included 1601 patients admitted between May 1, 1981 and
April 30, 1982.\\[2ex]
\underline{Variables include:} \\[2ex]
~~{\bf LOS} - Length of stay of a resident (in days)\\
~~{\bf AGE} - Age of a resident\\
~~{\bf RX} - Nursing home assignment (1:bonuses, 0:no bonuses)\\
~~{\bf GENDER} - Gender (1:male, 0:female)\\
~~{\bf MARRIED} - (1: married, 0:not married)\\
~~{\bf HEALTH} - health status (2:second best, 5:worst)\\
~~{\bf FAIL} - Failure/Censoring indicator (1:discharged,0:censored)
\\[2ex]
First few lines of data: \\[2ex]
37 86 1 0 0 2 0\\[2ex]
61 77 1 0 0 4 0\\[2ex]
%%%%%%%%%%%%%%%%%%%%%%%%%%%%%%%%%%%%%%%%%%%%%%%%%%%%%%%%%%%
\subsection{Fecundability}
Women who had recently given birth were asked to
recall how long it took them to become pregnant, and whether
or not they smoked during that time.
The outcome of interest is time
to pregnancy (in menstrual cycles).
\begin{center}
\begin{tabular}{ccc}
Cycle& Smokers & Non-smokers \\ \hline
1    &  29  &     198 \\
2    &  16  &     107 \\
3    &  17  &     ~55 \\
4    &  ~4  &     ~38 \\
5    &  ~3  &     ~18 \\
6    &  ~9  &     ~22 \\
7    &  ~4  &     ~~7 \\
8    &  ~5  &     ~~9 \\
9    &  ~1  &     ~~5 \\
10   &  ~1  &     ~~3 \\
11   &  ~1  &     ~~6 \\
12   &  ~3  &     ~~6 \\
12$+$&  ~7  &     ~12 \\ \hline
 \end{tabular}
 \end{center}
\normalsize
%%%%%%%%%%%%%%%%%%%%%%%%%%%%%%%%%%%%%%%%%%%%%%%%%%%%%%%%%%%
\subsection{MAC Prevention Clinical Trial}
ACTG 196 was a randomized clinical trial to study the effects of
combination regimens on prevention of MAC ({\em mycobacterium avium
complex}), one of the most common OIs in AIDS patients.\\[2ex]
\noindent
The {\bf treatment regimens} were:
%\vspace{-0.2in}
\begin{itemize}
\item clarithromycin (new)
\item rifabutin (standard)
\item clarithromycin plus rifabutin
\end{itemize}
Other characteristics of trial:
%\vspace{-0.2in}
\begin{itemize}
\item Patients enrolled between April 1993 and February 1994
\item Follow-up ended August 1995
\item In February 1994, rifabutin dosage was reduced from 3 pills/day
(450mg) to 2 pills/day (300mg) due to concern over {\bf uveitis}\footnote{\footnotesize
{\em Uveitis} is an adverse experience resulting in inflammation
of the uveal tract in the eyes (about 3-4\% of patients
reported uveitis).}
\end{itemize}
\noindent
The main intent-to-treat analysis compared the 3 treatment arms
without adjusting for this change in dosage.  \\[2ex]
%%%%%%%%%%%%%%%%%%%%%%%%%%%%%%%%%%%%%%%%%%%%%%%%%%%%%%%%%%%
\subsection{Example D: Time to first tuberculosis (TB) episode}
These data come from a longitudinal surveillance study of Kenyan children.  The data have
multiple lines per patient that correspond to multiple visits to the clinic.
Data gathered at each visit are:\\[2ex]
{\bf PATID} - Patient identification\\
{\bf timetotb} - Time from entry in the study until TB\\
{\bf first\_tb} - Whether this is the first TB episode\\
{\bf cd4} - Absolute CD4-positive lymphocyte count\\
{\bf cd4per} - CD4 percent\\
{\bf orphan} - Orphaned status\\
{\bf age} - Age (in years) at each visit\\[2ex]
The difference of this data from all the others is that the explanatory variables (e.g., CD4 count,
percent and so on) change over time.
{\bf More Definitions and Notation}\\[2ex]
There are several equivalent ways to characterize the
probability distribution of a survival random variable.
Some of these are familiar; others are special to survival
analysis.\\[2ex]
%\newpage
We will focus on the following terms:

\begin{itemize}
\item The density function $f(t)$
\item The survivor function $S(t)$
\item The hazard function $\lambda(t)$
\item The cumulative hazard function $\Lambda(t)$
\end{itemize}
\section{Survival distribution function}
\subsection{Density function (or Probability Mass Function) for
discrete r.v.'s}
Suppose that $T$ takes values in $a_1,a_2, \ldots, a_n$.
\begin{eqnarray*}
 f(t) & = &  Pr(T = t) \\[2ex]
 & = & \left\{ \begin{array}{ccc}
 f_j & \mbox{ if } & t=a_j, j=1,2,\ldots,n \\
 0 & \mbox{ if } & t\ne a_j, j=1,2,\ldots,n
 \end{array}
 \right.
 \end{eqnarray*}
\subsection{Density Function for continuous r.v.'s}
\[  f(t) = \lim_{\Delta t \rightarrow 0}
 \frac{1}{\Delta t}  Pr(t \le T \le t+\Delta t) \]
%%%%%%%%%%%%%%%%%%%%%%%%%%%%%%%%%%%%%%%%%%%%%%%%%%%%%%%%%%%
\subsection{Survivorship Function}
$ S(t) = P(T \ge t)$. \\[2ex]
In other settings, the cumulative distribution function, \newline
$F(t)=P(T\le t)$, is of interest.  In survival analysis, our
interest tends to focus on the survival function, $S(t)$.
\subsubsection{For a continuous random variable:}
\[   S(t) = \int_{t}^{\infty} f(u) du \]
\subsubsection{For a discrete random variable:}
\begin{eqnarray*}
S(t) & = & \sum_{u \ge t} f(u) \\[1ex]
     & = & \sum_{a_j \ge t} f(a_j) = \sum_{a_j \ge t} f_j
\end{eqnarray*}
\subsubsection{Notes:}
\begin{itemize}
\item  From the definition of $S(t)$ for a continuous variable,\\
$S(t)=1-F(t)$ as long as $f(t)$ is absolutely continuous
\item  For a discrete variable, we have to decide what to do if an
event occurs exactly at time $t$; i.e., does that become part of $F(t)$ or
$S(t)$?
\item To get around this problem, several books define \\
$S(t)=Pr(T>t)$, or else define $F(t)=Pr(T<t)$ \\(eg. Collett)
\end{itemize}
%%%%%%%%%%%%%%%%%%%%%%%%%%%%%%%%%%%%%%%%%%%%%%%%%%%%%%%%%%%
\section{Hazard Function $\lambda(t)$}
Sometimes called an {\em instantaneous failure
rate}, the {\em force of mortality}, or the
{\em age-specific failure rate}.

\subsection{Continuous random variables:}
\begin{eqnarray*}
\lambda(t)  & = &
\lim_{\Delta t \rightarrow 0} \, \frac{1}{\Delta t}\,
Pr(t \le T <  t+\Delta t | T \ge t) \\[2ex]
& = & \lim_{\Delta t \rightarrow 0} \, \frac{1}{\Delta t} \,
\frac{Pr([t \le T <  t+\Delta t] ~~ \bigcap~~ [T \ge t])}
{Pr(T \ge t)}\\[2ex]
& = & \lim_{\Delta t \rightarrow 0} \, \frac{1}{\Delta t} \,
\frac{Pr(t \le T <  t+\Delta t)}{Pr(T \ge t)}\\[2ex]
& = & \frac{f(t)}{S(t)}
\end{eqnarray*}
\subsection{Discrete  random variables:}
   \begin{eqnarray*}
  \lambda(a_j)\equiv \lambda_j   & = &  Pr(T = a_j | T \ge a_j) \\[1ex]
  & = & \frac{P(T=a_j)}{P(T\ge a_j)}\\[1ex]
  & = & \frac{f(a_j)}{S(a_j)}\\[1ex]
  & = & \frac{f(t)}{\sum_{k:a_k\ge a_j}f(a_k)}\\[1ex]
  \end{eqnarray*}
%%%%%%%%%%%%%%%%%%%%%%%%%%%%%%%%%%%%%%%%%%%%%%%%%%%%%%%%%%%
\section{Cumulative Hazard Function $\Lambda(t)$}
\subsection{Continuous random variables:}
\begin{eqnarray*}
\Lambda(t)  & = & \int_0^t \lambda(u) du
\end{eqnarray*}
\subsection{Discrete random variables:}
\begin{eqnarray*}
\Lambda(t) & = &  \sum_{k:a_k<t} \lambda_k
\end{eqnarray*}
The cumulative hazard does not have a very intuitive interpretation.
\\[2ex]
However, it turns out to be very useful for certain graphical
assessments:
\begin{itemize}
\item consistency with certain parametric models
\item evaluation of proportional hazards assumption for Cox models
\end{itemize}
%%%%%%%%%%%%%%%%%%%%%%%%%%%%%%%%%%%%%%%%%%%%%%%%%%%%%%%%%%%
\section{Relationship between $S(t)$ and $\lambda(t)$}

We've already shown that, for a continuous r.v.
$$\lambda(t)=\frac{f(t)}{S(t)}$$
For a left-continuous survivor function $S(t)$, we can show:
\begin{eqnarray*}
f(t) & = & - S'(t) ~~~~~\mbox{or}~~~ S'(t) ~ = ~ -f(t)
\end{eqnarray*}
We can use this relationship to show that:
\begin{eqnarray*}
- ~ \frac{d}{dt} [\log S(t)] & = &
- ~ \left(\frac{1}{S(t)}\right)\, S'(t)\\[1ex]
& = & - ~\frac{-f(t)}{S(t)} \\[1ex]
& = & \frac{f(t)}{S(t)}
\end{eqnarray*}
So another way to write $\lambda(t)$ is as follows:
\begin{eqnarray*}
\lambda(t) & = & - \frac{d}{dt} [\log S(t)]
\end{eqnarray*}
%%%%%%%%%%%%%%%%%%%%%%%%%%%%%%%%%%%%%%%%%%%%%%%%%%%%%%%%%%%
\section{Relationship between $S(t)$ and $\Lambda(t)$}
\subsection{Continuous case:}
\begin{eqnarray*}
\Lambda(t) & = & \int_0^t \lambda(u)du \\[1.5ex]
& = &  \int_0^t\frac{f(u)}{S(u)}du \\[1.5ex]
& = & \int_0^t -\frac{d}{du} \log S(u) du \\[1.5ex]
& = & -\log S(t)+\log S(0)\Rightarrow S(t)=e^{-\Lambda(t)}
\end{eqnarray*}
\subsection{Discrete case:} Suppose that $a_j<t\le a_{j+1}$.  Then
\begin{eqnarray*}
S(t) & = & P(T\ge a_1, T\ge a_2, \ldots, T\ge a_{j+1}) \\[1ex]
& = & P(T\ge a_1)P(T\ge a_2|T\ge a_1) \cdots
P(T\ge a_{j+1}|T\ge a_j)\\[1ex]
& = & (1-\lambda_1)\times \cdots \times (1-\lambda_j)\\[1ex]
& = & \prod_{k:a_k<t}(1-\lambda_k)
\end{eqnarray*}

Cox defines $\Lambda(t)=\sum_{k:a_k<t}\log(1-\lambda_k)$ so that
$S(t)=e^{-\Lambda(t)}$ in the discrete case, as well.
\section{Measuring Central Tendency in Survival}
\subsection{Mean survival}
Let's call this $\mu$ \
\begin{eqnarray*}
 \mu & = & \int_{0}^{\infty} u f(u)du ~~~\mbox{for continuous } T\\
 & = & \sum_{j=1}^n a_j f_j ~~~\mbox{for discrete } T
 \end{eqnarray*}
\subsection{Median survival}
The median survival, call this $\tau$, is defined by
\[  S(\tau) = 0.5 \]
Similarly, any other percentile could be defined.
\\[2ex]
In practice, we don't usually hit the median survival at exactly
one of the failure times.  In this case, the estimated median survival
is the {\em smallest} time $\tau$ such that $$\hat{S}(\tau)\leq 0.5$$
Some hazard shapes seen in applications:
\begin{itemize}
\item {\bf increasing}\\[1ex]
e.g. aging after 65
\item{\bf decreasing} \\[1ex]
e.g. survival after surgery
\item {\bf bathtub} \\[1ex]
e.g. age-specific mortality
\item {\bf constant} \\[1ex]
e.g. survival of patients with advanced chronic disease
\end{itemize}
%%%%%%%%%%%%%%%%%%%%%%%%%%%%%%%%%%%%%%%%%%%%%%%%%%%%%%%%%%%
\begin{center}
\section{Estimating the survival or hazard function}
\end{center}

We can estimate the survival (or hazard) function in two
ways:
\begin{itemize}
\item by specifying a parametric model for $\lambda(t)$
based on a particular density function $f(t)$
\item by developing an empirical estimate of the survival
function (i.e., non-parametric estimation)
\end{itemize}
\underline{If no censoring:}\\
The empirical estimate of the survival function, $\tilde{S}(t)$,
is the proportion of individuals with event times greater than $t$.
\\[2ex]
\underline{With censoring:}\\
If there are censored observations, then $\tilde{S}(t)$ is not a
good estimate of the true $S(t)$, so other non-parametric methods
must be used to account for censoring (life-table methods,
Kaplan-Meier estimator)
%%%%%%%%%%%%%%%%%%%%%%%%%%%%%%%%%%%%%%%%%%%%%%%%%%%%%%%%%%%
\section{Some Parametric Survival Distributions}
\subsection{Exponential distribution  (1 parameter)}
\begin{eqnarray*}
f(t) & = & \lambda e^{-\lambda t} \mbox{ for } t\ge 0 \\[2ex]
S(t) & = & \int_t^\infty f(u) du  =  e^{-\lambda t} \\[2ex]
\lambda(t) & = & \frac{f(t)}{S(t)} \\
& = & \lambda ~~~~~\mbox{ constant hazard!} \\[2ex]
\Lambda(t) & = & \int_0^t \lambda(u) \, du \\[1ex]
 & = & \int_0^t \lambda \, du = \lambda t
\end{eqnarray*}
{\bf Check:} Does $S(t)=e^{-\Lambda(t)}$ ? \\[2ex]
{\bf median:} solve $0.5=S(\tau)=e^{-\lambda \tau}$:
\[\Rightarrow \tau=\frac{-\log(0.5)}{\lambda}\]
{\bf mean:}
\[\int_0^\infty u\lambda e^{-\lambda u}du=\frac{1}{\lambda}\]
%%%%%%%%%%%%%%%%%%%%%%%%%%%%%%%%%%%%%%%%%%%%%%%%%%%%%%%%%%%
\subsection{\bf Weibull distribution (2 parameters)}
Generalizes exponential:
\begin{eqnarray*}
S(t) & = & e^{-\lambda t^\kappa} \\[2ex]
f(t) & = & \frac{-d}{dt}S(t)=\kappa \lambda t^{\kappa-1}
e^{-\lambda t^\kappa}\\[2ex]
\lambda(t) & = & \kappa \lambda t^{\kappa-1}\\[2ex]
\Lambda(t) & = & \int_0^t\lambda(u) du = \lambda t^\kappa
\end{eqnarray*}
\noindent
$\lambda$ -  the {\em scale} parameter, $\kappa$  - the {\em shape} parameter
\\[2ex]
The Weibull distribution is convenient because of simple
forms.  It includes several hazard shapes:\\[2ex]
$\kappa=1 \rightarrow \mbox{ constant hazard}$ \\[1ex]
$0<\kappa<1 \rightarrow \mbox{ decreasing hazard}$ \\[1ex]
$\kappa > 1 \rightarrow \mbox{ increasing hazard}$
\subsection{Rayleigh distribution}
Another 2-parameter generalization of exponential:
\[\lambda(t)=\lambda_0+\lambda_1 t\]
\subsection{Compound exponential}
\[T\sim \exp(\lambda), \, \lambda\sim g\]
\[ f(t)=\int_0^\infty \lambda e^{-\lambda t}g(\lambda)d\lambda\]
\subsection{Log-normal and log-logistic}
These are possible distributions for $T$ obtained by specifying for $\log T$
any convenient family of distributions, e.g.\\[1ex]
$\log T \sim$ normal (non-monotone hazard) \\[1ex]
$\log T \sim$ logistic
\subsection{Inverse Gaussian}
First passage time of Brownian motion to linear boundary.
\\[2ex]
Why use one versus another?
\begin{itemize}
\item technical convenience for estimation and inference\\[2ex]
\item explicit simple forms for $f(t), S(t)$, and $\lambda(t)$.\\[2ex]
\item qualitative shape of hazard function
\end{itemize}
One can usually distinguish between a one-parameter
model (like the exponential) and two-parameter (like Weibull or
log-Normal) in terms of the adequacy of fit to a dataset.
\\[2ex]
Without a lot of data, it may be hard to distinguish between
the fits of various 2-parameter models (i.e., Weibull vs log-normal)
\newpage
:
%%%%%%%%%%%%%%%%%%%%%%%%%%%%%%%%%%%%%%%%%%%%%%%%%%%%%%%%%%%
\end{document}
\newpage
\begin{center}
\fbox{\bf Preview of Coming Attractions}
\end{center}
\vspace{.1in}
Next class we will discuss the most famous non-parametric
approach for  estimating the survival distribution,
called the {\em Kaplan-Meier estimator}.

To motivate the derivation of this estimator, we will
first consider a set of survival times where there is
no censoring.

The following are {\bf times to relapse} (weeks) for 21 leukemia patients
receiving control treatment (Table 1.1 of Cox \& Oakes): \\[1.5ex]
\centerline{1, 1, 2, 2, 3, 4, 4, 5, 5, 8, 8, 8, 8, 11, 11, 12, 12, 15,
17, 22, 23}\\[2ex]
How would we estimate S(10), the probability that
an individual survives to time 10 or later?
\\[2ex]
What about $\tilde{S}(8)$?  ~~~~~ Is it $\frac{12}{21}$ or $\frac{8}{21}$?
\\[2ex]
Let's construct a table of $\tilde{S}(t)$:
\begin{center}
\begin{tabular}{cc}
Values of t   & ~~~~$\hat{S}(t)$~~~~ \\
\hline
$~~~~~ t \le 1 $  & 21/21=1.000 \\
$1 < t \le 2 $  & 19/21=0.905\\
$2 < t \le 3 $  & 17/21=0.809\\
$3 < t \le 4 $  & \\
$4 < t \le 5 $  & \\
$5 < t \le 8 $  & \\
$8 < t \le 11 $ &\\
$11 < t \le 12 $ &\\
$12 < t \le 15 $ &\\
$15 < t \le 17 $ &\\
$17 < t \le 22 $ &\\
$22 < t \le 23 $ &\\
\end{tabular}
\end{center}

\normalsize
\section{Empirical Survival Function:}
When there is no censoring, the general formula is:
\begin{eqnarray*}
\tilde{S}(t) = \frac{\#~individuals~ with~T \ge t}
       {total~sample~size}
\end{eqnarray*}
In most software packages, the survival function is evaluated
just after time $t$, i.e., at $t^{+}$.  In this case, we
only count the individuals with $T > t$.

%%%%%%%%%%%%%%%%%%%%%%%%%%%%%%%%%%%%%%%%%%%%%%%%%%%%%%%%%%%5
\subsection{Example for leukemia data (control arm)}
\centerline{\psfig{figure=surv_control.ps,height=3.5in}}

%%%%%%%%%%%%%%%%%%%%%%%%%%%%%%%%%%%%%%%%%%%%%%%%%%%%%%%%%%%5
\subsection{Stata Commands for Survival Estimation}
\small
\begin{verbatim}
.use leukem
.stset remiss status if trt==0       (to keep only untreated patients)
(21 observations deleted)
. sts list
         failure _d:  status
   analysis time _t:  remiss
           Beg.          Net    Survivor     Std.
  Time    Total   Fail   Lost   Function    Error     [95% Conf. Int.]
----------------------------------------------------------------------
     1       21      2      0     0.9048   0.0641     0.6700    0.9753
     2       19      2      0     0.8095   0.0857     0.5689    0.9239
     3       17      1      0     0.7619   0.0929     0.5194    0.8933
     4       16      2      0     0.6667   0.1029     0.4254    0.8250
     5       14      2      0     0.5714   0.1080     0.3380    0.7492
     8       12      4      0     0.3810   0.1060     0.1831    0.5778
    11        8      2      0     0.2857   0.0986     0.1166    0.4818
    12        6      2      0     0.1905   0.0857     0.0595    0.3774
    15        4      1      0     0.1429   0.0764     0.0357    0.3212
    17        3      1      0     0.0952   0.0641     0.0163    0.2612
    22        2      1      0     0.0476   0.0465     0.0033    0.1970
    23        1      1      0     0.0000        .          .         .
----------------------------------------------------------------------
.sts graph
\end{verbatim}
\end{document}
