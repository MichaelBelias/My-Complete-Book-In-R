In this lab, we are going to understand how to produce a Kaplan-Meier 
plot of survival estimates for more than one subgroup on the same graph, and 
compute a \emph{Logrank} or \emph{Wilcoxon} test in \verb|R|.
\begin{enumerate}[(a)]
 \item Using the following artificial data from two treatment groups
 \begin{align}
 \text{Group0:}\ 15, 18, 19, 19, 20 
 \quad \text{Group1:}\ 16^{+}, 18, 20^{+}, 23, 24^{+}, \nonumber
 \end{align}
 compare the survival of the two groups by computing the \emph{Logrank} test using the \verb|survdiff| function in \verb|R|. What's your conclusion? \textbf{Optional:} Write an \verb|R| program to construct tables like that on page 22 of today's notes.
 \item We are going to work on data from a leukemia remission study (Garrett 
1997). The data consist of 42
patients who are monitored over time to see how long (\emph{weeks}) it takes them to go out of remission
(\emph{remiss}: 1 = yes, 0 = no). Half of 
the patients received a new experimental drug and the other half received a 
standard drug (\emph{trt}: 1=6-MP, 0=Control). This dataset is called \emph{leukem}.csv. Import the data in \verb|R|. Use the \verb|factor| function to encode the \verb|trt| variable as a factor (please, see ?\verb|factor| for more details).
\begin{enumerate}[(i)]
\item Use the \verb|survfit| function to get the KM estimates in both groups. After using information from (a), guess what the syntax would be.
\item Plot the survival estimates of the two treatment groups using the 
\verb|plot| function. Which group seems to be doing better? Comment on the graph.
\item Compare the survival between the two groups in terms of the \emph{Logrank} and \emph{Wilcoxon} test using the \verb|survdiff| function. What do you conclude from these tests? Why do you think that there is a difference in the p-values of the two tests? Explain your opinion.
\item How could you informally check the proportional hazards assumption? Any ideas?
\end{enumerate}
 
\end{enumerate}